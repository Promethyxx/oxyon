\documentclass{article}

\usepackage[latin1]{inputenc}
\usepackage[T1]{fontenc}
\usepackage[francais]{babel}
\usepackage{makeidx}

\makeindex

\begin{document}

% Tir� des M�moires du duc de Luynes, Paris, Firmin-Didot 1863, t.I.

� la mort de M. le duc de
Bourgogne\index{Bourgogne@\textsc{Bourgogne}, Louis de France, duc
de}, lorsqu'il fut question d'aller jeter de l'eau b�nite, le feu Roi
d�cida que si les princes lorrains\index{Princes etrangers@Princes
�trangers} s'y presenteroient, qu'eux ni les ducs n'en jetteroient ;
mais que si MM. de Rohan\index{Rohan@\textsc{Rohan}, duc de} et de
Bouillon y �toient, les ducs\index{Ducs et pairs} jetteroient de l'eau
b�nite avant eux : ce qui arriva effectivement ; mais MM. de Rohan et
de Bouillon\index{Bouillon|see{Princes �trangers}}, voyant les ducs
passer avant eux, s'en all�rent. Ce qui avoit �t� d�cid� en faveur de
MM. les ducs fut �crit sur le registre\index{Ceremonies@C�r�monies,
service des!registres} de M. de Dreux\index{Ceremonies@C�r�monies,
service des} ; mais deux ans apr�s, les repr�sentations de Mme de
Maintenon d�termin�rent le Roi � faire un changement et � ordonner �
M. de Dreux que cet article seroit ray� sur le registre. Il fut mis en
marge que le Roi n'avoit jamais voulu d�cider entre les ducs et MM. de
Rohan et de Bouillon.

\printindex

\end{document}



